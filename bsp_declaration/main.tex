\documentclass[11pt,a4paper,english]{article}

\usepackage[
backend=biber,
style=numeric,
sorting=ynt
]{biblatex}

\addbibresource{references.bib}

\title{PKEX protocol Analysis}
\date{September 2024}
\author{
  Meireles Lopes, Steve\\
  \texttt{steve.meireles.001@student.uni.lu}\\
  \texttt{022148763b}
  \and
  Skrobot, Marjan\\
  \texttt{marjan.skrobot@uni.lu}
}

\begin{document}
\maketitle
Secondary Language: German
\section{Abstract}
In the current day and age IoT is increasingly growing. Most people have
interacted with IoT devices in some form. While smartphones are the most
prominent IoT devices, many others exist, such as smart fridges,
temperature sensors, IP cameras, and more. The more our environment is getting
smart, the more people are concerned about the security behind these devices
and ask themselves if they really should equip themselves with more devices
that could potentially get hacked. This raises the question: What security
protocols are used in the communication between IoT devices? Are these
protocols secure by design or are there improvements that could be made? To
answer these questions, the paper is going to look at the solution "Wi-Fi Easy
Connect" from the WiFi
Alliance\cite{easyconnect}\cite{alliance}\cite{wiki_alliance} which is a
worldwide network of companies that are responsible for a lot of the
certification programs that are used by most of the population for example
Wi-Fi 6\cite{wifi6}. The objective of this paper is to analyze the protocols
used in the "Wi-Fi Easy Connect" solution with a particular focus on the
protocol PKEX which employs a shared Code, Key, Phrase, or Word to establish a
secure connection between the communicating devices.

\printbibliography
\end{document}
