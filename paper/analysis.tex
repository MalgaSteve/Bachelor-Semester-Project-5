\section{Analysis}
\subsection{Detailed Definition}
As described above PKEX has two phases.
\subsubsection{Authentication phase}
The next steps are done by both parties, Alice and Bob.
Values known to each party: \(pw, Pi, Pr\)
Alice picks random \(x\) and calculates \(X, Q\_a, M\):
\[x, X = x * G\]
\[Q\_a = H(pw) * Pi\]
\[M = X + Q\_a\]

Bob picks random \(y\) and calculates \(Y, Q\_b\):
\[y, Y = y * G\]
\[Q\_b  = H(pw) * Pr\]

Alice sends \(Alice, M\) to Bob. Bob derives values and generates an ephemeral
secret key.
\[Q\_a = H(pw) * Pi\]
\[X' = M - Q\_a\]
\[N = Y + Q\_b\]
\[z = KDF-n(F(y*X', Alice | Bob | F(M) | F(N) | pw))\]

Bob sends \(Bob, N\) to Bob and Alice derives values and an ephemeral secret key.
\[Q\_b = H(pw) * Pr\]
\[Y' = N - Q\_b\]
\[z = KDF-n(F(x*Y', Alice | Bob | F(M) | F(N) | pw))\]

\subsubsection{Reveal phase}
Here \(a = F(B)\) is a function that takes an element and returns a scalar.
Alice picks random a and calculates A, then derives u. 
\[a, A = a * G \]
\[u = HMAC(F(a * Y'), Alice | F(A) | F(Y') | F(X))\]
Alice sends \(A, u\) using authentication encryption with the first half of the key
naming it \(z\_0\). Bob checks the encryption and derives values. 
\[if (AE-decrypt returns fail) fail\]
\[if (A not valid element) fail\]
\[u' = HMAC(F(y * A), Alice | F(A) | F(Y) | F(X'))\]
\[if (u' != u) fail\]
\[v = HMAC(F(b * X'), Bob | F(B) | F(X') | F(Y))\]
Bob sends \(B, v\) using authentication encryption with the second half of the key
naming it \(z\_1\). Alice checks the encryption and derives values. 
\[if (AE-decrypt returns fail) fail\]
\[if (B not valid element) fail\]
\[v = HMAC(F(x * B), Bob | F(B) | F(X) | F(Y'))\]
\[if (v' != v) fail\]

\subsection{Limitations}

PKEX has several limitations. First, the security of the protocol has not yet
been formally proven. There are several reasons for this. One reason, as
mentioned by the author, is that he is not an academic expert and may not have
the resources or expertise to conduct formal proofs. However, the protocol's
design has inherent challenges to proving its security. Specifically, the reuse
of ephemeral keys in both the authentication and reveal phases complicates the
proof for several reasons:

\begin{itemize} 
	\item The reuse of ephemeral keys \begin{itemize}

	\item the constructor \_\_init\_\_() which initializes the all the

		necessary values 

	\item start() which generates the ephemeral public key

	\item finish(inbound\_message) which authenticates the opposite entity and

		generates an ephemeral secret key

\end{itemize}creates dependencies between the two
		phases, making it more difficult to prove their security independently.
		Ideally, each phase should be isolated so that the security proof can
		be constructed without accounting for interactions between the phases.
	\item This dependency increases the potential for unforeseen
		vulnerabilities, as the interactions between the phases may introduce
		complex attack vectors. As a result, constructing and verifying a
		formal proof of security becomes significantly harder. 
	\item The reuse of ephemeral keys might expose patterns in the two phases
		that an attacker could exploit, leading to the potential leakage of
		private key information. Consequently, the proof must account for how
		the Proof of Possession (PoP) guarantees hold in such scenarios.
\end{itemize}

The lack of modularity in PKEX further limits its flexibility. If the protocol
were structured in a more modular and independent way, each phase could be
modified or improved without impacting the others. Additionally, if a
vulnerability were discovered in one phase, it could be replaced or updated
without affecting the overall system. However, the current design ties the
phases together, making it difficult to implement changes or fixes without
disrupting the entire protocol.
