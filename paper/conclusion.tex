\section{Conclusion}

In conclusion, while PKEX \cite{harkins-pkex-06} is widely used and included in
the Easy-Connect Specification, it exhibits several limitations. This paper
thoroughly analyzes the PKEX protocol, identifies its shortcomings, and
proposes an improved solution to address these issues.

The primary limitations of PKEX highlighted in this work are the lack of
modularity and the reuse of ephemeral keys, which could impact both flexibility
and security. The PKEX protocol is structured in two phases: the authentication
phase and the reveal phase. The first phase employs the SPAKE2 protocol
proposed by Abdalla \cite{Abdalla_2005}, while the second utilizes values from
the initial phase to achieve proof of possession and bind a private key to an
entity.

Our proposed protocol addresses these limitations by ensuring that the second
phase does not reuse values directly from the first phase, instead relying
only on the output of the authentication phase. This approach improves
modularity and security. In the authentication phase, any secure PAKE
protocol such as SPAKE2 can be used. Similarly, any proof-of-possession
mechanism such as the Schnorr Signature can be integrated into the second
phase. This modularity allows for greater flexibility and the potential to
adopt newer cryptographic primitives as they emerge.

To validate our proposal, we implemented and compared both protocols
\cite{malga_steve}. Benchmarking results show that while our protocol is
slightly slower for single exchanges, it is faster when performing multiple
exchanges, demonstrating its practical advantages in scenarios requiring
repeated key exchanges.

\subsubsection{Future Improvements}

While our proposal demonstrates clear benefits, certain areas remain open for
improvement. The lack of a formal proof of security for the protocol should be
addressed, as it would provide stronger theoretical guarantees than PKEX.
Additionally, the current implementation could be further refined to enhance
modularity and ease of integration with other cryptographic components.
