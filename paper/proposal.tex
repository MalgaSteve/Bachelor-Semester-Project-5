\section{Our Proposal}

Our proposal aims to address the limitations identified in the PKEX
protocol by proposing a more modular and flexible approach. Our protocol
maintains the same security properties as PKEX while allowing for greater
adaptability and ease of implementation.

\subsection{Properties}
Our proposal preserves the core security properties of PKEX:
\begin{itemize}
	\item An adversary cannot subvert the exchange without knowing the password.
	\item An adversary cannot obtain the password through a passive attack.
	\item The protocol detects whether a guess of the password is correct.
	\item Proof of possession of the private key is ensured.
	\item At the end of the exchange, trust is established in the entity's
		public key, which is cryptographically bound to the entity's identity.
		The exchange fails if this binding is not confirmed.
\end{itemize}

\subsection{Notation}
For clarity, we introduce the following notation:
\begin{itemize}
	\item Password or shared secret key: \( pw \)
	\item Private key: \( sk \)
	\item Public key: \( pk \)
\end{itemize}

\subsection{Protocol Definition}
Our protocol follows a two-phase structure, similar to PKEX, but with the
flexibility to use different PAKE protocols and modular components.

\subsubsection{Authentication Phase}
In this phase, the protocol is very similar to the authentication phase in
PKEX. However, unlike PKEX, which uses SPAKE2, our proposal allows using 
any secure Password-Authenticated Key Exchange (PAKE) protocol. This provides
greater flexibility, enabling the use of more secure or efficient PAKE
protocols as needed. 

Let’s assume two parties: Alice and Bob. They engage in a PAKE protocol to
derive a shared secret and public key. The result is the following:

\[
\text{Alice} \xrightarrow{\text{PAKE}} \text{Bob} \rightarrow (\text{shared secret } sk, \text{public key } pk)
\]

\subsubsection{Proof of Possession Phase}
In the second phase, the Proof of Possession (PoP) mechanism is incorporated to
verify the ownership of the private key corresponding to the public key
exchanged in the previous phase. This is achieved by having the entities prove
they possess the private key without revealing it. This ensures that only the
entity with the correct private key can complete the exchange, providing
additional security against impersonation and man-in-the-middle attacks.

The protocol's modularity allows for the replacement or enhancement of
individual components, such as the PAKE or PoP methods, without requiring
changes to the entire protocol. This design makes our protocol more adaptable
and easier to update with newer cryptographic techniques as they become
available.

Formal Definition:

\begin{definition} [Key generation and proof of possession scheme] A key
	generation and proof of possession (KGPOP) scheme consists of: 
\begin{itemize}
	\item \(PoP.KG() \rightarrow (pk, sk)\):A probabilistic key
		generation algorithm that outputs a public, secret key pair.
	\item $PoP.PG(pk, sk, attrs)\rightarrow \pi$: A probabilistic proof
		generation algorithm that takes as input a public key, secret key, and
		attributes $attrs \in \{0, 1\}^*$, and outputs a proof string. (In our
		application, $attrs$ could be the body of a certificate signing
		request, for example.) In this case could be binding id with PAKE.
    
    \item $PoP.Vf(pk,attrs,\pi)\rightarrow \{0,1\}$: A deterministic verification algorithm that takes as input a public key $pk$, attributes $attrs$, and proof $\pi$, and outputs $1$ if the proof is valid, and $0$ otherwise.
    
\end{itemize}
\end{definition}

\subsection{Example implementation}
In our implementation, we decided to use SPAKE2 as PAKE protocol, and for the
proof of possession, we use Schnorr signature.

\subsubsection{SPAKE2 Implementation} For the implementation we use the
implementation of Warner \cite{warner_spake2}. The implementation provides a
spake2.py file. Inside the file, there are several classes. We use the class
"SPAKE2\_A" which has following main methods.

\begin{itemize}
	\item the constructor \_\_init\_\_() which initializes the all the
		necessary values 
	\item start() which generates the ephemeral public key
	\item finish(inbound\_message) which authenticates the opposite entity and
		generates an ephemeral secret key
\end{itemize}

Code Snippet:
\begin{minted}[frame=lines, linenos, breaklines]{python}
	class SPAKE2_Base:
    "This class manages one side of a SPAKE2 key negotiation."

    side = None # set by the subclass

    def __init__(self, password,
                 params=DefaultParams, entropy_f=os.urandom):
        assert isinstance(password, bytes)
        self.pw = password
        self.pw_scalar = params.group.password_to_scalar(password)

        assert isinstance(params, _Params), repr(params)
        self.params = params
        self.entropy_f = entropy_f

        self._started = False
        self._finished = False

    def start(self):
        if self._started:
            raise OnlyCallStartOnce("start() can only be called once")
        self._started = True

        g = self.params.group
        self.xy_scalar = g.random_scalar(self.entropy_f)
        self.xy_elem = g.Base.scalarmult(self.xy_scalar)
        self.compute_outbound_message()
        # Guard against both sides using the same side= by adding a side byte
        # to the message. This is not included in the transcript hash at the
        # end.
        outbound_side_and_message = self.side + self.outbound_message
        return outbound_side_and_message

    def finish(self, inbound_side_and_message):
        if self._finished:
            raise OnlyCallFinishOnce("finish() can only be called once")
        self._finished = True

        g = self.params.group
        inbound_elem = g.bytes_to_element(self.inbound_message)
        if inbound_elem.to_bytes() == self.outbound_message:
            raise ReflectionThwarted
        #K_elem = (inbound_elem + (self.my_unblinding() * -self.pw_scalar)
        #          ) * self.xy_scalar
        pw_unblinding = self.my_unblinding().scalarmult(-self.pw_scalar)
        K_elem = inbound_elem.add(pw_unblinding).scalarmult(self.xy_scalar)
        K_bytes = K_elem.to_bytes()
        key = self._finalize(K_bytes)
        return key

class SPAKE2_A(SPAKE2_Asymmetric):
	side = SideA
	def my_blinding(self): return self.params.M
	def my_unblinding(self): return self.params.N
	def X_msg(self): return self.outbound_message
	def Y_msg(self): return self.inbound_message

class SPAKE2_B(SPAKE2_Asymmetric):
	side = SideB
	def my_blinding(self): return self.params.N
	def my_unblinding(self): return self.params.M
	def X_msg(self): return self.inbound_message
	def Y_msg(self): return self.outbound_message
\end{minted}

\subsubsection{POP}
For the proof of possession, the Schnorr Signature is used. 
The POP phase consists of two parts. 

The first part consists of key generation and signing:
We use a hash key derivation function which we then split into three
keys. 
\[sk1, sk2, sk3 = HKDF(ephermal\_key)\]
Pick random scalar a (or b) and Compute A (or B)
\[A = G*a\]
or
\[B = G*b\]
Pick random scalar k and compute K:
\[K = G*k\]

Compute e and s such that:
\[e = H(K || A (or B) || PakeID)\]
\[s = k + e * (-a (or b))\]

Outbound message to send:
\[(A, s, e)\]
or
\[(B, s, e)\]

The second part consists of verification:
Upon receiving the previous output you have:
\[A', s', e'\]
Re-calculate e:
\[e'' = H(K || A (or B) || PakeID)\]
Output:
\[True, if e''== e'\]
\[False, if otherwise\]

Code snippet:
\begin{minted}[frame=lines, linenos, breaklines]{python}
class I_PKEX(SPAKE2_Asymmetric):

    def start_pkex(self, key):
        if not self._finished: 
            raise RunSpakeFirst("start_pkex() may only be called when SPAKE2 protocol is finished running")

        g = self.params.group

        sk = HKDF(
                algorithm=hashes.SHA256(),
                length=len(key)*3,
                salt=None,
                info=None,
                ).derive(key)

        split_size = len(sk) // 3
        sk1 = sk[:split_size]
        assert len(sk1) == split_size, len(sk1)
        sk2 = sk[split_size:2*split_size]
        assert len(sk2) == split_size, len(sk2)
        self.pake_id = sk[2*split_size:]
        assert len(self.pake_id) == split_size, len(self.pake_id)

        # inbound_element is public key A (pA)
        # inbound_elem = g.bytes_to_element(self.inbound_message) 

        # sk = a, pk = g * a = A
        self.ab_scalar = g.random_scalar(self.entropy_f)
        self.AB_element = g.Base.scalarmult(self.ab_scalar)

        # random k
        self.k = g.random_scalar(self.entropy_f)
        # print("ab_scalar: ", self.ab_scalar)
        # print("k", self.k)
        self.K_element = g.Base.scalarmult(self.k)

        self.e = hashes.Hash(hashes.SHA256())
        self.e.update(self.K_element.to_bytes() + self.AB_element.to_bytes() + self.pake_id)
        self.e = self.e.finalize()

        print("To be checked: ", self.e)
        self.e = g.password_to_scalar(self.e)
        
        print(self.k)
        self.s = (self.k + self.e * (-self.ab_scalar)) % L

        return (self.AB_element.to_bytes(), self.s, self.e)

    def finalize(self, key, data):
        g = self.params.group

        (AB_element, s, e) = data
        AB_element = g.bytes_to_element(AB_element) 

        g_s = g.Base.scalarmult(s)
        pk_e = AB_element.scalarmult(e)

        K_element_check = g_s.add(pk_e)
        e_check = hashes.Hash(hashes.SHA256())
        e_check.update(K_element_check.to_bytes() + AB_element.to_bytes() + self.pake_id)
        e_check = e_check.finalize()
        e_check = g.password_to_scalar(e_check)
        print("To be checked: ", e)
        print("Checking e: ", e_check)

        return e_check == e
\end{minted}

\subsection{Benchmarks}
\begin{table}[h!]
\centering
\begin{tabular}{|c|c|c|}
\hline
Benchmarks & Average time for one exchange \\ \hline
PKEX   & 0,03060 seconds\\ \hline
Improved PKEX   & 0,03666\\ \hline
\end{tabular}
\caption{Average time for one exchange}
\label{tab:basic}
\end{table}

\begin{table}[h!]
\centering
\begin{tabular}{|c|c|c|}
\hline
Benchmarks & Average time for one exchange \\ \hline
PKEX   & 0,35223 seconds\\ \hline
Improved PKEX   & 0,31128\\ \hline
\end{tabular}
\caption{Average time for 10 exchanges}
\label{tab:basic}
\end{table}

While the improved PKEX has a slightly higher cost for a single exchange maybe
due to its modular components. It has a better performance over multiple
exchanges. The reason may be because of the optimizations in key reuse.
